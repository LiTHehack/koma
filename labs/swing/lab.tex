\documentclass[11pt,a4paper]{article}
\pdfoutput=1

\usepackage[utf8x]{inputenc}
\usepackage[T1]{fontenc}
\usepackage[swedish]{babel}
\usepackage{amsmath}
\usepackage{lmodern}
\usepackage{units}
\usepackage{icomma}
\usepackage{color}
\usepackage{graphicx}
\usepackage{multicol,caption}
\usepackage{bbm}
\usepackage{hyperref}
\usepackage{xfrac}
\newcommand{\N}{\ensuremath{\mathbbm{N}}}
\newcommand{\Z}{\ensuremath{\mathbbm{Z}}}
\newcommand{\Q}{\ensuremath{\mathbbm{Q}}}
\newcommand{\R}{\ensuremath{\mathbbm{R}}}
\newcommand{\C}{\ensuremath{\mathbbm{C}}}
\newcommand{\rd}{\ensuremath{\mathrm{d}}}
\newcommand{\id}{\ensuremath{\,\rd}}

\begin{document}
\thispagestyle{empty}
\pagestyle{empty}

\title{TNM040 Kommunikation och användargränssnitt\\Laboration 1}

\begin{center}
  \LARGE TNM040 Kommunikation och användargränssnitt\\Laboration 1
\end{center}

\section*{Översikt}
Den här laborationen behandlar främst menyer i Java. Tanken är att du här ska bygga vidare på den sista labben (miniprojektet) från kursen TND002 OOP. Uppgiften består kort sagt i att byta ut det GUI som du gjorde då mot ett menybaserat GUI.

Om du har kvar dina filer från TND002 så kan du skippa resten av detta avsnitt och gå direkt till upppgiften, detta oavsett om du gjorde standardprojektet i form av ett telefonregister eller om du gjorde ett eget projekt med annat tema.

Annars kan du kopiera arkivfilen \texttt{code.zip} från labb-mappen. Den innehåller \texttt{class}-filer för klasserna \texttt{Person} och \texttt{Register} samt dokumentation för båda klasserna. Dessutom ingår det ett enkelt testprogram \texttt{Test.java} som visar hur de båda klasserna kan användas.

Om ``gamla'' JDK 6.0 är installerat på datorn ifråga (och så är fallet i vissa datorsalar) så måste du använda \texttt{class}-filerna från underkatalogen \texttt{jdk6}.

\section*{Uppgift}
Konstruera ett nytt menybaserat GUI för telefonregistret (eller motsvarande tema). Hur detta gränssnitt ska utformas avgör du (bäst) själv men ditt program ska även omfatta följande saker:

\begin{itemize}
  \item kortkommandon för några menyval.
  \item en popup-meny med några menyval.
  \item fildialoger i samband med att registret öppnas eller sparas. Se t.ex. Skansholm avsnitt 14.7 angående klassen \texttt{JFileChooser}.
  \item flera innerklasser för händelsehantering. Alltså ingen \texttt{actionPerformed} direkt i GUI-klassen.
\end{itemize}

Och det är förstås tillåtet att lägga till fler finesser.

\section*{Redovisning}

Demonstrera dina program för assistenten. Någon skriftlig redovisning behövs inte.

\end{document}
